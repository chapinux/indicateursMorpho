\documentclass[[a4paper, 11pt]{article}
\usepackage{graphicx}
\usepackage[T1]{fontenc}
\usepackage[utf8]{inputenc}
\usepackage[french]{babel}
\usepackage{amsmath}
\usepackage{lmodern}
\begin{document}
%
\title{Indicateurs, morphologiques pour la plupart}
%
%\titlerunning{Abbreviated paper title}
% If the paper title is too long for the running head, you can set
% an abbreviated paper title here
%
\author{Team SIMPLU3D}
%

%
\maketitle              % typeset the header of the contribution
%
\begin{abstract}
Mesures récoltées dans différents papiers conseillés par Mickaël ou trouvés sur Internet, dans la revue Building and Environment, EPB, etc...	  
\end{abstract}
%
%
%
\section{Compacités}

La compacité $C_f$ d'une forme $f$ (forme = objet tridimensionnel plein) est définie comme le rapport de son volume $V_f$  et de sa surface extérieure $A_f$

\begin{equation}
C_f= \frac{V_f}{A_f}
\end{equation}



\subsection{Compacités relatives} % (fold)
\label{subsec:compacités_relatives}
Issu de \cite{pessenlehner_building_2003}


La compacité relative $RC(f)$ d'une forme $f$ est le ratio de sa compacité et de la compacité d'une forme de référence «pertinente», de même volume $V_{ref}=V_f$: 
\begin{itemize}
	\item sphère :  forme de compacité maximale, irréaliste pour un bâtiment
	\item demi-sphère : plus réaliste car «faisable» (cf. igloo)
	\item cube : plus réaliste encore étant donnée la forme «carrée »des bâtiments modernes occidentaux. 
\end{itemize}


\begin{equation}
RC_f= \frac{V_f}{A_f}*\frac{A_{ref}}{V_{ref}}
\end{equation}

avec $V_{ref}=V_f$. (ça revient donc au ratio des aires de la forme et de la forme de référence de volume équivalent)

L'article  \cite{pessenlehner_building_2003} donne directement les formules suivantes:


\begin{equation}
RC_{sphère}(f) \approx 4.84 * V_f^{2/3} * A_f^{-1}
\end{equation}


\begin{equation}
RC_{cube}(f) =6 * V_f^{2/3} * A_f^{-1}
\end{equation}


\begin{equation}
RC_{hemisphère}(f) \approx 3.83 * V_f^{2/3} * A_f^{-1}
\end{equation}


Détail du calcul de $RC_{sphère}$ :

On cherche l'aire $A_{ref}$ de la sphère de volume équivalent à la forme $f$ ($V_{ref}=V_f$.)
Soit $\rho$ le rayon de cette sphère. 

De $V_{ref}=V_f=\frac{4\pi}{3}\rho^3 $, on déduit $\rho$:

\begin{equation*}
\rho = \left (\frac{3}{4\pi}\right)^{1/3} *(V_f)^{1/3}
\end{equation*}


d'où:
\begin{align*}
A_{ref} &= 4\pi * \rho^2\\
 &= 4\pi*\left( \frac{3}{4\pi}\right)^{2/3}*(V_f)^{2/3}
\end{align*}
et 
\begin{align*}
RC_{sphère}(f)&= \frac{A_{ref}}{A_f}\\
       &=4\pi*\left( \frac{3}{4\pi}\right)^{2/3}*(V_f)^{2/3} * A_f^{-1}\\
       &\approx 4.84 * V_b^{2/3} * A_f^{-1} 
\end{align*}


\textbf{Valeurs:} Théoriquement $RC\in[0;1]$ , plus c'est proche de 1 , plus la forme est proche de celle de référence. Sans unité.

% section compacités_relatives (end)




\subsection{Pour l'énergie et le climat} % (fold)
\label{sub:pour_l_énergie_et_le_climat}

% subsection pour_l_énergie_et_le_climat (end)
D'après \cite{pessenlehner_building_2003}, la compacité relative est corrélée avec la charge de chauffage $HL$ (\textbf{\textit{heating load}}) et la surchauffe (\textit{overheating}) des bâtiments.
Les auteurs insistent sur le fait que la RC ne capture pas suffisament les propriétés de la morphologie du bâtiment  pour que le lien avec l'overheating soit fort (il manque l'ombrage par exemple)

\begin{equation}
HL = 105*RC_{cube}^{-0.75} * V^{-0.25} 
\end{equation}

L'ajustement statistique est obtenu sur des simus thermiques de 12 bâtiments formés de 18 cubes, dont l'assemblage varie pour obtenir des compacités différentes (de 0.62 à 0.98 avec un volume de référence cubique)  du chauffage des bâtiments en faisant varier la surface vitrée des murs, l'orientation du bâtiment: 720 combinaisons simulées. 




\begin{equation}
x + y = z
\end{equation}
Please try to avoid rasterized images for line-art diagrams and
schemas. Whenever possible, use vector graphics instead (see

%\begin{figure}
%\includegraphics[width=\textwidth]{fig1.eps}
%\caption{A figure caption is always placed below the illustration.
%Please note that short captions are centered, while long ones are
%justified by the macro package automatically.} \label{fig1}
%\end{figure}


\section{Sky View Factor} % (fold)
\label{sec:sky_view_factor}


C'est une mesure qui donne , en un point , la fraction de ciel visible par rapport à celle d'un hémisphère centré en ce point.
(peut être que c'est un rapport entre les deux angles solides, à vérifier avec qqn qui sait )



\begin{equation}
 V_{sky} \approx \frac{1}{2\pi}\int_0^{2\pi}[cos \beta sin^2 H_{\phi} + sin \beta cos(\phi -\alpha)*(H_{\phi}-sinH_{\phi} cosH_{\phi}) ]d\phi	
\end{equation}


\textbf{Valeurs} $\in [0;1]$ , sans unité.
Plus c'est roche de 0 , plus c'est bouché, moins il y a de ciel.



\section{Coverage Ratio}

Le coverage ratio est simplement le ratio entre la somme des aires au sol des bâtiments ($A_i$)d'une zone et l'aire de la zone $A_z$

\begin{equation}
	\lambda_p =  \frac{1}{A_z} \sum_{i=1}^n A_i
\end{equation}

\textbf{Valeurs} $\in [0;1]$ , sans unité



% section sky_view_factor (end)

%
% ---- Bibliography ----
%
% BibTeX users should specify bibliography style 'splncs04'.
% References will then be sorted and formatted in the correct style.
%
% \bibliographystyle{splncs04}

\bibliography{morpho}
\end{document}
