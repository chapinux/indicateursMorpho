\documentclass[[a4paper, 11pt]{article}
\usepackage{graphicx}
\usepackage[utf8]{inputenc}
\usepackage[french]{babel}

\begin{document}
%
\title{Indicateurs, morphologiques pour la plupart}
%
%\titlerunning{Abbreviated paper title}
% If the paper title is too long for the running head, you can set
% an abbreviated paper title here
%
\author{Team SIMPLU3D}
%

%
\maketitle              % typeset the header of the contribution
%
\begin{abstract}
Mesures récoltées dans différents papiers conseillés par Mickaël ou trouvés sur Internet, dans la revue Building and Environment, EPB, etc...	  
\end{abstract}
%
%
%
\section{Compacité}

La compacité $C_f$ d'une forme $f$ (forme = objet tridimensionnel plein) est définie comme le rapport de son volume $V_f$  et de sa surface extérieure $S_f$

\begin{equation}
C_f= \frac{V_f}{S_f}
\end{equation}



\subsection{Volume de référence}



\section{Compacités relatives} % (fold)
\label{sec:compacités_relatives}

% section compacités_relatives (end)


D'après \cite{pessenlehner_building_2003}, la compacité relative est corrélée avec le heating load et la surchauffe (\textit{overheating}) des bâtiments.
Les ajustements suivants sont obtenus sur la base de simulation à grains fins du chauffage des bâtiments en faisant varier leur  





\subsubsection{Sample Heading (Third Level)} Only two levels of
headings should be numbered. Lower level headings remain unnumbered;
they are formatted as run-in headings.

\paragraph{Sample Heading (Fourth Level)}
The contribution should contain no more than four levels of
headings. Table~\ref{tab1} gives a summary of all heading levels.

\begin{table}
\caption{Table captions should be placed above the
tables.}\label{tab1}
\begin{tabular}{|l|l|l|}
\hline
Heading level &  Example & Font size and style\\
\hline
Title (centered) &  {\Large\bfseries Lecture Notes} & 14 point, bold\\
1st-level heading &  {\large\bfseries 1 Introduction} & 12 point, bold\\
2nd-level heading & {\bfseries 2.1 Printing Area} & 10 point, bold\\
3rd-level heading & {\bfseries Run-in Heading in Bold.} Text follows & 10 point, bold\\
4th-level heading & {\itshape Lowest Level Heading.} Text follows & 10 point, italic\\
\hline
\end{tabular}
\end{table}


\noindent Displayed equations are centered and set on a separate
line.
\begin{equation}
x + y = z
\end{equation}
Please try to avoid rasterized images for line-art diagrams and
schemas. Whenever possible, use vector graphics instead (see
Fig.~\ref{fig1}).

\begin{figure}
\includegraphics[width=\textwidth]{fig1.eps}
\caption{A figure caption is always placed below the illustration.
Please note that short captions are centered, while long ones are
justified by the macro package automatically.} \label{fig1}
\end{figure}




%
% ---- Bibliography ----
%
% BibTeX users should specify bibliography style 'splncs04'.
% References will then be sorted and formatted in the correct style.
%
% \bibliographystyle{splncs04}

\bibliography{morpho}
\end{document}
